
{\chapter{Using the Definitions in the Left Direction}}

{\bf{1.}} Suppose you have some $T(n)$, for example $T(n) = 3n^2 - n$
which you know is less than or equal to $3n^2$ for all
$n \ge 0$~\cite{whole-journal}.
Using the definition of $O()$ in the $\Leftarrow$ direction,
we have $g(n) = n^2$, $c = 3$, and $n_0 = 0$ so that $T(n) = O(n^2)$.

\vspace*{2em}

{\bf{2.}} Suppose you have another $T(n)$,
for example $T(n) = 5n{\cdot}\log{n} + 42$
which you know is less than or equal to
$5n{\cdot}\log{n}$
for all $n \ge 0$.
Using the definition of $\Omega()$ in the $\Leftarrow$ direction,
we have $g(n) = n{\cdot}\log{n}$,
$c = 5$, and $n_0 = 0$ so that $T(n) = \Omega(n{\cdot}\log{n})$.

\vspace*{2em}

{\bf{3.}} Suppose you have a third $T(n)$
which you know is in {\bf{both}} $O(n)$ and $\Omega(n)$.
Using the definition of $\Theta()$ in the $\Leftarrow$ direction,
we have $T(n) = \Theta(n)$.


